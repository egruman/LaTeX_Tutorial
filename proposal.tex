\documentclass{article}
\usepackage[utf8]{inputenc}
\usepackage[margin=1in]{geometry}
\usepackage{charter}

% *********** color and links ************
\usepackage{xcolor}
\usepackage[nottoc]{tocbibind} % add bibliography to toc
\usepackage[colorlinks = true,
            linkcolor = black,
            citecolor = gray]{hyperref}
\hypersetup{
    linktoc=all, %set to all if you want both sections and subsections linked
}
\usepackage{hyperref}
% ****************************************

\title{\normalsize 76-270 Writing for the Professions \\ \LARGE Instructional Manual Proposal}
\author{Ethan Gruman \and Cody Johnson \and Taisuke Yasuda}

\begin{document}

\maketitle

\section{Outline of instructional manual}
\subsection{Problem statement}
Our group will be working on an instructional manual on how to use \LaTeX\ for beginners. This instructional manual will fulfill the needs of freshman math and computer science students of getting started on typesetting their assignmnets in \LaTeX. This skill is required of students enrolled in the required course Concepts of Mathematics (course number 21-128 and 15-151) and many classes following it. Although the TAs of this course teach \LaTeX, they do not have a standardized source to our knowledge, and the instructions are disseminated through a recitation that the TAs hold. Because of this, many students struggle to learn \LaTeX\ for the first time. Beyond the context of Carnegie Mellon University, this instructional manual would benefit anyone who wants to get started on using \LaTeX. We will continue researching for the scope and quality of the need within Carnegie Mellon University by asking freshman math and computer science students. We anticipate that this instructional manual will give these students an organized document that they can refer to when they are first getting started.

\subsection{Audience}
As hinted above, our primary audience is the freshman class in the math and computer science departments. As secondary audiences, we will keep anyone who wants to learn \LaTeX\ in mind. There are about 50 freshman in the math department and 50 freshman in the computer science department. All three of us have some friends in this audience group, and they would at least be willing to be involved in some user testing.

\subsection{Purpose statement}
The overall goal of this instructional manual is to get a total beginner to start creating basic documents in \LaTeX\ with math mode and basic external packages. Some of the main subtasks that we currently have in mind are
\begin{enumerate}
  \item creating an account on \url{v2.overleaf.com}
  \item modifying some basic fields in the template provided by overleaf
  \item formatting mathematical expressions
  \item using basic external packages
\end{enumerate}
Our purpose ensures that our instructions will be nontrivial, as many students struggle to learn \LaTeX\ for the first time. Our purpose ensures that our instructions will be novel, since these instructions are only available through TAs at the time of writing.

\section{Goals and expectations}
The success of the group is the success for each of the three of us. As a group, we will deem ourselves successful if we can help even one concepts student learn \LaTeX. Cody and Tai have very similar strengths, in that they are both extremely advanced users of \LaTeX, and also have plentiful experience in teaching students, through serving as graders and teaching assistants. Ethan brings a unique perspective to the group as a beginner-level user of \LaTeX\ who can bring insight into what beginners struggle with. Each of us are busy which can be an obstacle, but we expect for our obstacles to be nonoverlapping, allowing at least one of us to make progress on the project at every time. If one of us misses a deadline or gives poor-quality contributions, we will cover each other and move on, with the expectation that they make it up later on.

\section{Schedule}
\begin{center}
\begin{tabular}{l c c l}
  Task & Responsible Member & Reviewing Member & Due Date \\
  \hline
  Proposal & All & All & 11/12 \\
  Overleaf tutorial & Ethan & Cody & 11/15\\
  Structure of a \LaTeX\ document & Cody & Tai & 11/17 \\
  Group discussion & All & All & 11/19 \\
  Basic modifications & Tai & Ethan & 11/22 \\
  Formatting math & Ethan & Cody & 11/24 \\
  Group discussion & All & All & 11/26 \\
  External packages & Cody & Tai & 11/28 \\
  Additional resources & Tai & Ethan & 11/28 \\
  Group discussion & All & All & 12/03 \\
  Polishing & All & All & 12/05 \\
  Final Deadline & All & All & 12/09
\end{tabular}
\end{center}

\end{document}
